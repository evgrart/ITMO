\documentclass[14pt,final,oneside]{article}% класс документа, характеристики
%>>>>> Разметка документа
\usepackage[a4paper, mag=1000, left=1cm, right=1cm, top=1cm, bottom=1cm, headsep=0.7cm, footskip=1cm]{geometry} % По ГОСТу: left>=3cm, right=1cm, top=2cm, bottom=2cm,
\linespread{1} % межстройчный интервал по ГОСТу := 1.5
%<<<<< Разметка документа
\usepackage[utf8]{inputenc}
\usepackage[T2A]{fontenc}
\usepackage[english, russian]{babel}
\usepackage{amsmath, amsfonts, amssymb, amssymb, amsthm, mathtools}
\usepackage{geometry}
\pagenumbering{gobble} % откл нумерацию
\usepackage{colortbl} % таблички
\usepackage{listings} % листинг
\usepackage{dcolumn} % выравнивание чисел
\usepackage[normalem]{ulem} % для подчёркиваний uline
\ULdepth = 0.16em % расстояние от линии до текста выше/ниже
\lstset{basicstyle=\ttfamily\normalsize}  
\usepackage{graphicx}
\usepackage{xcolor}
\usepackage{float}
\lstset{
    language=Java,
    numbers=left,                    % Нумерация строк слева
    stepnumber=1,                    % Каждая строка нумеруется
    numbersep=5pt,                   % Отступ от кода до номеров строк
    showspaces=false,                % Не показывать пробелы
    showstringspaces=false,          % Не показывать пробелы в строках
    tabsize=4,                       % Размер табуляции
    breaklines=true,                 % Перенос строк
    breakatwhitespace=false,         % Переносить строки по пробелам
    basicstyle=\ttfamily,            % Шрифт текста
    keywordstyle=\color{blue},       % Цвет ключевых слов
    commentstyle=\color{green},      % Цвет комментариев
    stringstyle=\color{red},         % Цвет строковых литералов
}
\usepackage{titlesec}
\titleformat{\section}
{\LARGE\bfseries}
{\thesection}{15pt}{} 
\usepackage{fancyhdr}
\theoremstyle{theorem}
\newtheorem{theorem}{Теорема}
\newtheorem{lemma}{Лемма}
\renewcommand{\thesection}{\arabic{section}.}
\renewcommand{\thesubsection}{\arabic{section}.\arabic{subsection}.}
\usepackage{tocloft}
\renewcommand{\cftsecleader}{\cftdotfill{\cftdotsep}}
\renewcommand{\cftsecfont}{\Large\bfseries} 
\renewcommand{\cfttoctitlefont}{\LARGE\bfseries}
\newcommand{\lr}[1]{\left( {#1} \right)}
\usepackage{hyperref} % гиперссылки
\hypersetup{
    colorlinks=true,        % Включаем цветные ссылки
    linkcolor=blue!50!black, % Цвет внутренних ссылок (например, на разделы)
    urlcolor=blue!50!black,  % Цвет URL-ссылок
    citecolor=blue!50!black, % Цвет ссылок на библиографию
}


\begin{document}
\noindent \href{https://www.youtube.com/watch?v=QsBYshN5idw}{\Large{Алгоритм работы кода Хэмминга}} \\
\textbf{1.} Чем классический код Хэмминга отличается от неклассического кода Хэмминга? \\
Код Хэмминга — блочный равномерный разделимый самокорректирующийся код. Исправляет одиночные битовые ошибки, возникшие при передаче или хранении данных. \\
Расширенный код Хэмминга: иногда к классическому коду Хэмминга добавляют ещё один проверочный бит, что позволяет не только исправлять одиночные ошибки, но и обнаруживать двойные ошибки. В этом случае код уже называется расширенным кодом Хэмминга. \\
\textbf{2.} Необходимо передать 20 информационных бит. Каким классических кодом Хэмминга необходимо воспользоваться? Чем будут заполнены оставшиеся информационные биты? \\
Классический код Хэмминга исправляет одиночные ошибки, и количество проверочных битов r должно удовлетворять следующему неравенству: $2^r \geq r + i + 1$, где i — количество информационных бит (в данном случае 20), r — количество проверочных бит. Подходит $r = 6$, в сообщении будут 20 информационнных бит без изменений и 6 проверочных. \\
\textbf{3.} В результате выполнения некоторого алгоритма коэффициент сжатия получился разным 0,05. Что это означает? \\
Сжатие данных — процесс, обеспечивающий уменьшение объёма данных путём сокращения их избыточности. Если коэффициент сжатия получился равным 0,05, то сжатый файл занимает 5\% от исходного размера. \\
\textbf{4.} Чем контрольная сумма отличается от бита чётности? \\
Контрольная сумма и бит чётности — это два разных механизма для обнаружения ошибок в данных. Контрольная сумма — некоторое число, рассчитанное путем применения определенного алгоритма к набору данных и используемое для проверки целостности этого набора данных при их передаче или хранении. Бит чётности — частный случай контрольной суммы, представляющий из себя 1 контрольный бит, используемый для проверки четности количества единичных битов в двоичном числе. \\
\textbf{5.} Для чего нужны различные способы обработки блоков данных, полученных с ошибкой в результате передачи? \\
Различные способы обработки блоков данных с ошибками необходимы для того, чтобы обеспечить баланс между надёжностью передачи, скоростью передачи данных, и ресурсами, затрачиваемыми на исправление или перезапрос ошибок. Выбор метода зависит от конкретных условий передачи данных, критичности информации и ресурсов системы. Например, важные данные, такие как банковские транзакции, требуют абсолютной точности, тогда как небольшие ошибки в потоках видео или аудио могут быть незначительными. \\
\textbf{6.} Что такое запрещённые комбинации? \\
В некоторых системах кодирования часть комбинаций битов может быть зарезервирована для специальных целей, таких как синхронизация или служебные команды. Эти комбинации называются запрещёнными, так как они не могут использоваться для представления обычных данных. Например, в кодах с исправлением ошибок (например, код Хэмминга), некоторые комбинации могут указывать на ошибку или на факт того, что кодовое слово повреждено (если среди $s_i$ есть значение 1). \\
\textbf{7.} Чем отличается коэффициент сжатия от коэффициента избыточности? \\
Коэффициент сжатия — отношение размера входного потока к выходному потоку. Коэффициент избыточности — отношение числа проверочных разрядов (r) к общему числу разрядов ($n = i + r$).


\end{document}


