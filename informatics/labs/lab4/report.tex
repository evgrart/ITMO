\documentclass[14pt,final,oneside]{article}% класс документа, характеристики
%>>>>> Разметка документа
\usepackage[a4paper, mag=1000, left=3cm, right=1.5cm, top=2cm, bottom=2cm, headsep=0.7cm, footskip=1cm]{geometry} % По ГОСТу: left>=3cm, right=1cm, top=2cm, bottom=2cm,
\linespread{1} % межстройчный интервал по ГОСТу := 1.5
%<<<<< Разметка документа
\usepackage[utf8]{inputenc}
\usepackage[T2A]{fontenc}
\usepackage[english, russian]{babel}
\usepackage{amsmath, amsfonts, amssymb, amssymb, amsthm, mathtools}
\usepackage{geometry}
\usepackage{colortbl} % таблички
\usepackage{listings} % листинг
\usepackage{dcolumn} % выравнивание чисел
\usepackage[normalem]{ulem} % для подчёркиваний uline
\ULdepth = 0.16em % расстояние от линии до текста выше/ниже
\lstset{basicstyle=\ttfamily\normalsize}  
\usepackage{graphicx}
\usepackage{xcolor}
\usepackage{float}
\lstset{
    language=Java,
    numbers=left,                    % Нумерация строк слева
    stepnumber=1,                    % Каждая строка нумеруется
    numbersep=5pt,                   % Отступ от кода до номеров строк
    showspaces=false,                % Не показывать пробелы
    showstringspaces=false,          % Не показывать пробелы в строках
    tabsize=4,                       % Размер табуляции
    frame=single
    breaklines=true,                 % Перенос строк
    breakatwhitespace=false,         % Переносить строки по пробелам
    basicstyle=\ttfamily,            % Шрифт текста
    keywordstyle=\color{blue},       % Цвет ключевых слов
    commentstyle=\color{green},      % Цвет комментариев
    stringstyle=\color{red},         % Цвет строковых литералов
}
\usepackage{titlesec}
\titleformat{\section}
{\LARGE\bfseries}
{\thesection}{15pt}{} 
\usepackage{fancyhdr}
\theoremstyle{theorem}
\newtheorem{theorem}{Теорема}
\newtheorem{lemma}{Лемма}
\renewcommand{\thesection}{\arabic{section}.}
\renewcommand{\thesubsection}{\arabic{section}.\arabic{subsection}.}
\usepackage{tocloft}
\renewcommand{\cftsecleader}{\cftdotfill{\cftdotsep}}
\renewcommand{\cftsecfont}{\Large\bfseries} 
\renewcommand{\cfttoctitlefont}{\LARGE\bfseries}
\newcommand{\lr}[1]{\left( {#1} \right)}
\usepackage{hyperref} % гиперссылки
\hypersetup{
    colorlinks=true,        % Включаем цветные ссылки
    linkcolor=blue!50!black, % Цвет внутренних ссылок (например, на разделы)
    urlcolor=blue!50!black,  % Цвет URL-ссылок
    citecolor=blue!50!black, % Цвет ссылок на библиографию
}

\begin{document}
\thispagestyle{empty}
\begin{center}
\LARGE{Университет ИТМО} 
\vspace{20pt}

\LARGE{Факультет программной инженерии и компьютерной техники \\
Образовательная программа системное и прикладное программное обеспечение}
\vspace{160pt}

\LARGE{Лабораторная работа  \textnumero 4 \\
По дисциплине "Информатика" \\ 
Вариант 26}
\vspace{120pt}
\end{center}

\begin{flushright}
\LARGE{Выполнил студент группы P3109 \\
Евграфов Артём Андреевич \\
Проверил:\\
Рыбаков Степан Дмитриевич}
\vspace{120pt}
\end{flushright}

\begin{center}
\Large{Санкт-Петербург 2024}
\end{center}

\newpage
\setcounter{page}{1}
\tableofcontents
\newpage
\section{Задание}
\underline{Обязательное задание} (позволяет набрать до 45 процентов от максимального числа баллов БаРС за данную лабораторную): написать программу на языке Python 3.x или любом другом, которая бы осуществляла парсинг и конвертацию исходного файла в новый путём простой замены метасимволов исходного формата на метасимволы результирующего формата. \\
Нельзя использовать готовые библиотеки, в том числе регулярные выражения в Python и библиотеки для загрузки XML-файлов. \\
\underline{Дополнительное задание №1} (позволяет набрать +10 процентов от максимального числа баллов БаРС за данную лабораторную). \\
a) Найти готовые библиотеки, осуществляющие аналогичный парсинг и конвертацию файлов. \\
b) Переписать исходный код, применив найденные
библиотеки. Регулярные выражения также нельзя
использовать. \\
c) Сравнить полученные результаты и объяснить их
сходство/различие. Объяснение должно быть отражено в
отчёте. \\
\underline{Дополнительное задание №2} (позволяет набрать +10 процентов от максимального числа баллов БаРС за данную лабораторную). \\
a) Переписать исходный код, добавив в него использование регулярных выражений. \\
b) Сравнить полученные результаты и объяснить их сходство/различие. Объяснение должно быть отражено в отчёте. \\
\underline{Дополнительное задание №3} (позволяет набрать +25 процентов от максимального числа баллов БаРС за данную лабораторную). \\
а) Переписать исходный код таким образом, чтобы для решения задачи использовались формальные грамматики. То есть ваш код должен уметь осуществлять парсинг и конвертацию любых данных, представленных в исходном формате, в данные, представленные в результирующем формате: как с готовыми библиотеками из дополнительного задания №1. \\
b) Проверку осуществить как минимум для расписания с двумя учебными днями по два занятия в каждом. \\
с) Сравнить полученные результаты и объяснить их сходство/различие. Объяснение должно быть отражено в отчёте. \\
\underline{Дополнительное задание №4} (позволяет набрать +5 процентов от максимального числа баллов БаРС за данную лабораторную). \\
a) Используя свою исходную программу из обязательного задания и программы из дополнительных заданий, сравнить стократное время выполнения парсинга + конвертации в цикле. \\
b) Проанализировать полученные результаты и объяснить их сходство/различие. Объяснение должно быть отражено в отчёте. \\
\underline{Дополнительное задание №5} (позволяет набрать +5 процентов от максимального числа баллов БаРС за данную лабораторную). \\
a) Переписать исходную программу, чтобы она осуществляла парсинг и конвертацию исходного файла в любой другой формат (кроме JSON, YAML, XML, HTML): PROTOBUF, TSV, CSV, WML и т.п \\
b) Проанализировать полученные результаты, объяснить особенности использования формата. Объяснение должно быть отражено в отчёте.
\section{Исходные файлы}
\href{https://github.com/evgrart/ITMO/blob/main/informatics/labs/lab4/schedule.json}{JSON файл} \\
\href{https://github.com/evgrart/ITMO/blob/main/informatics/labs/lab4/schedule.yaml}{YAML файл} \\
\href{https://github.com/evgrart/ITMO/blob/main/informatics/labs/lab4/schedule.toml}{TOML файл} \\
\section{Обязательное задание}
Решение задания по \href{https://github.com/evgrart/ITMO/blob/main/informatics/labs/lab4/task0.py}{ссылке} \\
JSON (JavaScript Object Notation) — это текстовый формат, используемый для представления структурированных данных. JSON легко читается человеком и одновременно легко обрабатывается компьютерами. \\
YAML (YAML Ain't Markup Language) — это текстовый формат, предназначенный для сериализации (процесс преобразования данных или объектов в формат, который можно легко сохранить или передать, а затем восстановить в исходном виде) данных. YAML разработан для удобства человека, чтобы писать и читать конфигурационные файлы и обмениваться данными между системами.
\section{Дополнительное задание \textnumero 1}
Решение задания по \href{https://github.com/evgrart/ITMO/blob/main/informatics/labs/lab4/task1.py}{ссылке} \\
Парсинг выполнен с помощью библиотек json и yaml. Различий между результатом работы программы предыдущего задания и текущего нет, поскольку оба парсера для данного JSON файла полностью эквивалентны.
\section{Дополнительное задание \textnumero 2}
Решение задания по \href{https://github.com/evgrart/ITMO/blob/main/informatics/labs/lab4/task2.py}{ссылке} \\
Различий между результатом работы программы
предыдущего задания и текущего нет, поскольку множество строк подходящих под условия (if-ы) из обязательного задания совпадает с множеством строк, подходящих под регулярные выражения из данного задания. 
\section{Дополнительное задание \textnumero 4}
Решение задания по \href{https://github.com/evgrart/ITMO/blob/main/informatics/labs/lab4/task4.py}{ссылке} \\
\begin{figure}[H]
    \centering
    \includegraphics[width=1\linewidth]{P1(L4I).png}
    \caption{Результат работы программы}
\end{figure}
\noindent Дольше всего работает программа с готовыми библиотеками, так как она реализует полноценный парсинг с полной обработкой входных и выходных файлов. Код с регулярными выражениями работает немного дольше основного кода, поскольку сами регулярные выражения довольно медленные, однако разница незначительна из-за малых размеров читаемого файла.
\section{Дополнительное задание \textnumero 5}
Парсер из JSON в TOML доступен по \href{https://github.com/evgrart/ITMO/blob/main/informatics/labs/lab4/test5.py}{ссылке} \\
TOML (Tom's Obvious, Minimal Language) — это язык разметки, который используется для настройки и конфигурации программ. Его основное назначение — предоставлять простой, человекочитаемый формат для структурированных данных. TOML особенно популярен в средах, где требуется легко настраивать параметры приложений или передавать конфигурации между системами. \\
Конфигурационные файлы — это текстовые файлы, которые используются для настройки программного обеспечения или операционных систем. Они позволяют разработчикам и пользователям задавать параметры работы приложения, не изменяя его исходный код. 
\section{Вывод}
Во время выполнения лабораторной работы я узнал о языках разметки JSON, TOML и YAML, научился с ними работать и переводить один тип данных в другой с помощью встроенных средств языка Python. Научился работать с библиотекой timeit для измерения времени и некоторыми библиотеками для анализа этих форматов и автоматического парсинга между ними. 
\section{Список литературы}
Балакшин П.В., Соснин В.В., Калинин И.В., Малышева Т.А., Раков С.В., Рущенко Н.Г., Дергачев А.М.
Информатика: лабораторные работы и тесты: Учебно-методическое пособие / Рецензент: Поляков
В.И. - Санкт-Петербург: Университет ИТМО, 2019. - 56 с. - экз. - Режим доступа: \\
\href{https://books.ifmo.ru/catalog/2019/catalog\_2024.htm}{https://books.ifmo.ru/catalog/2019/catalog\_2024.htm} \\

\noindent Лямин А.В., Череповская Е.Н. Объектно-ориентированное программирование. Компьютерный практикум. – СПб: Университет ИТМО, 2017. – 143 с. – Режим доступа: \\ \href{https://books.ifmo.ru/file/pdf/2256.pdf}{https://books.ifmo.ru/file/pdf/2256.pdf}
\end{document}