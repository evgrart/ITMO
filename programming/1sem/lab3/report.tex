\documentclass[14pt,final,oneside]{article}% класс документа, характеристики
%>>>>> Разметка документа
\usepackage[a4paper, mag=1000, left=3cm, right=1.5cm, top=2cm, bottom=2cm, headsep=0.7cm, footskip=1cm]{geometry} % По ГОСТу: left>=3cm, right=1cm, top=2cm, bottom=2cm,
\linespread{1} % межстройчный интервал по ГОСТу := 1.5
%<<<<< Разметка документа
\usepackage[utf8]{inputenc}
\usepackage[T2A]{fontenc}
\usepackage[english, russian]{babel}
\usepackage{amsmath, amsfonts, amssymb, amssymb, amsthm, mathtools}
\usepackage{geometry}
\usepackage{colortbl} % таблички
\usepackage{listings} % листинг
\usepackage{dcolumn} % выравнивание чисел
\usepackage[normalem]{ulem} % для подчёркиваний uline
\ULdepth = 0.16em % расстояние от линии до текста выше/ниже
\lstset{basicstyle=\ttfamily\normalsize}  
\usepackage{graphicx}
\usepackage{xcolor}
\usepackage{float}
\lstset{
    language=Java,
    numbers=left,                    % Нумерация строк слева
    stepnumber=1,                    % Каждая строка нумеруется
    numbersep=5pt,                   % Отступ от кода до номеров строк
    showspaces=false,                % Не показывать пробелы
    showstringspaces=false,          % Не показывать пробелы в строках
    tabsize=4,                       % Размер табуляции
    frame=single
    breaklines=true,                 % Перенос строк
    breakatwhitespace=false,         % Переносить строки по пробелам
    basicstyle=\ttfamily,            % Шрифт текста
    keywordstyle=\color{blue},       % Цвет ключевых слов
    commentstyle=\color{green},      % Цвет комментариев
    stringstyle=\color{red},         % Цвет строковых литералов
}
\usepackage{titlesec}
\titleformat{\section}
{\LARGE\bfseries}
{\thesection}{15pt}{} 
\usepackage{fancyhdr}
\theoremstyle{theorem}
\newtheorem{theorem}{Теорема}
\newtheorem{lemma}{Лемма}
\renewcommand{\thesection}{\arabic{section}.}
\renewcommand{\thesubsection}{\arabic{section}.\arabic{subsection}.}
\usepackage{tocloft}
\renewcommand{\cftsecleader}{\cftdotfill{\cftdotsep}}
\renewcommand{\cftsecfont}{\Large\bfseries} 
\renewcommand{\cfttoctitlefont}{\LARGE\bfseries}
\newcommand{\lr}[1]{\left( {#1} \right)}
\usepackage{hyperref} % гиперссылки
\hypersetup{
    colorlinks=true,        % Включаем цветные ссылки
    linkcolor=blue!50!black, % Цвет внутренних ссылок (например, на разделы)
    urlcolor=blue!50!black,  % Цвет URL-ссылок
    citecolor=blue!50!black, % Цвет ссылок на библиографию
}

\begin{document}
\thispagestyle{empty}
\begin{center}
\LARGE{Университет ИТМО} 
\vspace{20pt}

\LARGE{Факультет программной инженерии и компьютерной техники \\
Образовательная программа системное и прикладное программное обеспечение}
\vspace{160pt}

\LARGE{Лабораторная работа  \textnumero 3-4 \\
По дисциплине "Программирование" \\ 
Вариант 23445}
\vspace{120pt}
\end{center}

\begin{flushright}
\LARGE{Выполнил студент группы P3109 \\ 
Евграфов Артём Андреевич \\
Преподаватель: \\
Мустафаева Айнур Вугар кызы}
\vspace{120pt}
\end{flushright}

\begin{center}
\Large{Санкт-Петербург 2024}
\end{center}

\newpage
\setcounter{page}{1}
\tableofcontents
\newpage

\section{Задание}
Текст задания:
\begin{figure}[H]
    \centering
    \includegraphics[width=1\linewidth]{P1(L3-4P).png}
\end{figure}

\section{Исходный код}
Исходный код можно найти по ссылке на \href{https://github.com/evgrart/ITMO/tree/main/programming/1sem/lab3}{гитхабе}

\section{UML диаграмма}
\begin{figure}[H]
    \centering
    \includegraphics[width=1\linewidth]{P2(L3-4P).png}
\end{figure}

\section{Вывод программы}
Гостиная: здесь можно поиграть в крестики-нолики и посмотреть телевизор. Возможная побочка: смена настроения. \\
Сейчас в LivingRoom находятся: \\
Betan \\
Bosse \\
Betan и Bosse играют в крестики-нолики: \\
\[
\begin{array}{c|c|c}
   &   &   \\
\hline
   & X &   \\
\hline
   &   &  
\end{array}
\]
\[
\begin{array}{c|c|c}
   &   &   \\
\hline
   & X &   \\
\hline
   &   & O 
\end{array}
\]
\[
\begin{array}{c|c|c}
   &   &   \\
\hline
   & X &   \\
\hline
   & X & O 
\end{array}
\]
\[
\begin{array}{c|c|c}
 O &   &   \\
\hline
   & X &   \\
\hline
   & X & O 
\end{array}
\]
\[
\begin{array}{c|c|c}
 O &   &   \\
\hline
   & X & X \\
\hline
   & X & O 
\end{array}
\]
\[
\begin{array}{c|c|c}
 O &   &   \\
\hline
   & X & X \\
\hline
 O & X & O 
\end{array}
\]
\[
\begin{array}{c|c|c}
 O &   &   \\
\hline
 X & X & X \\
\hline
 O & X & O 
\end{array}
\]
Игрок Betan победил! \\
Теперь Betan имеет состояние HAPPY. \\
Теперь Bosse имеет состояние SAD. \\

\noindent Спальня: здесь можно поспать, чтобы улучшилось настроение и восстановились силы. \\
Сейчас в Bedroom находятся: \\
Mother \\
Mother пошёл (пошла) спать. \\
Работяга Mother не выспался (выспалась). \\
Теперь Mother имеет состояние ANGRY. \\
Здоровье персонажа Mother стало равным 185 (уменьшилось на 5). \\
Mother сейчас смотрит телевизор. Он(а) включил(а) "Секреты мастерства": советы по рукоделию, кулинарии и творчеству. \\
Теперь Mother имеет состояние HAPPY. \\

\noindent Кухня: любимое место в доме. Здесь можно подкрепиться (если холодильник не пустой) и восстановить хп, если здоровье достаточно низкое. \\
Сейчас в Kitchen находятся: \\
Malish \\
Malish любуется цветами, расставленными на кухне. \\
Теперь Malish имеет состояние HAPPY. \\
Malish съел(а) из холодильника Pizza, состояние улучшилось. \\
Здоровье персонажа Malish стало равным 155 (повысилось на 5). \\
Неожиданно настроение Malish резко изменилось. \\
Теперь Malish имеет состояние SAD. \\
Местоположение персонажа Betan с LivingRoom сменилось на Kitchen. \\
Betan съел(а) из холодильника Soup, состояние улучшилось. \\
Здоровье персонажа не было изменено, поскольку и так довольно высокое. \\

\noindent Произошёл несчастный случай! \\
Здоровье персонажа Malish стало равным 63 (уменьшилось на 92). \\
Реакция Betan на здоровье Malish: \\
Давай поиграем, когда ты станешь лучше! Я уже придумал игру. \\
Теперь Betan имеет состояние NEUTRAL. \\
Реакция Bosse на здоровье Malish: \\
Это так печально... Надеюсь, он(а) скоро поправится. \\
Malish жалуется Mother на состояние здоровья. \\
Реакция Mother на здоровье Malish: \\
Malish, что с тобой? Это нехорошо. Схожу за первой помощью. \\
Здоровье персонажа Malish стало равным 78 (повысилось на 15). \\
Mother уже вызвал(а) скорую помощь для Malish. \\
Травмы незначительные - скорая помощь выехала! Приедет через 66 минут. \\
Скорая прибыла! \\
Здоровье персонажа Malish стало равным 98 (повысилось на 20). \\
Скорая выехала в госпиталь. \\
Malish прибыл(а) в госпиталь. Надеемся, он(а) скоро поправится. \\

\section{Вывод}
В ходе данной лабораторной работы я: \\
Познакомился принципами проектирования SOLID; \\
Ознакомился с понятиями абстрактного класса, интерфейса и перечисляемых типов; \\
Улучшил свои навыки составления UML диаграмм; \\
Познакомился с исключениями, их иерархией, и научился создавать собственные исключения;

\end{document}